\documentclass[a4paper,14pt]{extreport} % формат документа

\usepackage{amsmath}
\usepackage{cmap} % поиск в ПДФ
\usepackage[T2A]{fontenc} % кодировка
\usepackage[utf8]{inputenc} % кодировка исходного текста
\usepackage[english,russian]{babel} % локализация и переносы
\usepackage[left = 2cm, right = 1cm, top = 2cm, bottom = 2 cm]{geometry} % поля
\usepackage{listings}
\usepackage{graphicx} % для вставки рисунков
\usepackage{amsmath}
\usepackage{float}
\usepackage{longtable}
\usepackage{multirow}
\graphicspath{{pictures/}}
\DeclareGraphicsExtensions{.pdf,.png,.jpg}
\newcommand{\anonsection}[1]{\section*{#1}\addcontentsline{toc}{section}{#1}}

\lstset{ %
	language=Prolog,                % Язык программирования 
	numbers=left,                   % С какой стороны нумеровать          
	frame=single,                    % Добавить рамку
	 escapebegin=\begin{russian}\commentfont,
    escapeend=\end{russian},
    literate={Ö}{{\"O}}1
    {Ä}{{\"A}}1
    {Ü}{{\"U}}1
    {ß}{{\ss}}1
    {ü}{{\"u}}1
    {ä}{{\"a}}1
    {ö}{{\"o}}1
    {~}{{\textasciitilde}}1
    {а}{{\selectfont\char224}}1
    {б}{{\selectfont\char225}}1
    {в}{{\selectfont\char226}}1
    {г}{{\selectfont\char227}}1
    {д}{{\selectfont\char228}}1
    {е}{{\selectfont\char229}}1
    {ё}{{\"e}}1
    {ж}{{\selectfont\char230}}1
    {з}{{\selectfont\char231}}1
    {и}{{\selectfont\char232}}1
    {й}{{\selectfont\char233}}1
    {к}{{\selectfont\char234}}1
    {л}{{\selectfont\char235}}1
    {м}{{\selectfont\char236}}1
    {н}{{\selectfont\char237}}1
    {о}{{\selectfont\char238}}1
    {п}{{\selectfont\char239}}1
    {р}{{\selectfont\char240}}1
    {с}{{\selectfont\char241}}1
    {т}{{\selectfont\char242}}1
    {у}{{\selectfont\char243}}1
    {ф}{{\selectfont\char244}}1
    {х}{{\selectfont\char245}}1
    {ц}{{\selectfont\char246}}1
    {ч}{{\selectfont\char247}}1
    {ш}{{\selectfont\char248}}1
    {щ}{{\selectfont\char249}}1
    {ъ}{{\selectfont\char250}}1
    {ы}{{\selectfont\char251}}1
    {ь}{{\selectfont\char252}}1
    {э}{{\selectfont\char253}}1
    {ю}{{\selectfont\char254}}1
    {я}{{\selectfont\char255}}1
    {А}{{\selectfont\char192}}1
    {Б}{{\selectfont\char193}}1
    {В}{{\selectfont\char194}}1
    {Г}{{\selectfont\char195}}1
    {Д}{{\selectfont\char196}}1
    {Е}{{\selectfont\char197}}1
    {Ё}{{\"E}}1
    {Ж}{{\selectfont\char198}}1
    {З}{{\selectfont\char199}}1
    {И}{{\selectfont\char200}}1
    {Й}{{\selectfont\char201}}1
    {К}{{\selectfont\char202}}1
    {Л}{{\selectfont\char203}}1
    {М}{{\selectfont\char204}}1
    {Н}{{\selectfont\char205}}1
    {О}{{\selectfont\char206}}1
    {П}{{\selectfont\char207}}1
    {Р}{{\selectfont\char208}}1
    {С}{{\selectfont\char209}}1
    {Т}{{\selectfont\char210}}1
    {У}{{\selectfont\char211}}1
    {Ф}{{\selectfont\char212}}1
    {Х}{{\selectfont\char213}}1
    {Ц}{{\selectfont\char214}}1
    {Ч}{{\selectfont\char215}}1
    {Ш}{{\selectfont\char216}}1
    {Щ}{{\selectfont\char217}}1
    {Ъ}{{\selectfont\char218}}1
    {Ы}{{\selectfont\car219}}1
    {Ь}{{\selectfont\char220}}1
    {Э}{{\selectfont\char221}}1
    {Ю}{{\selectfont\char222}}1
    {Я}{{\selectfont\char223}}1
    {і}{{\selectfont\char105}}1
    {ї}{{\selectfont\char168}}1
    {є}{{\selectfont\char185}}1
    {ґ}{{\selectfont\char160}}1
    {І}{{\selectfont\char73}}1
    {Ї}{{\selectfont\char136}}1
    {Є}{{\selectfont\char153}}1
    {Ґ}{{\selectfont\char128}}1
}

\begin{document}
\begin{titlepage}

    \begin{table}[H]
        \centering
        \footnotesize
        \begin{tabular}{cc}
            \multirow{8}{*}{\includegraphics[scale=0.35]{bmstu.jpg}}
            & \\
            & \\
            & \textbf{Министерство науки и высшего образования Российской Федерации} \\
            & \textbf{Федеральное государственное бюджетное образовательное учреждение} \\
            & \textbf{высшего образования} \\
            & \textbf{<<Московский государственный технический} \\
            & \textbf{университет имени Н.Э. Баумана>>} \\
            & \textbf{(МГТУ им. Н.Э. Баумана)} \\
        \end{tabular}
    \end{table}

    \vspace{-2.5cm}

    \begin{flushleft}
        \rule[-1cm]{\textwidth}{3pt}
        \rule{\textwidth}{1pt}
    \end{flushleft}

    \begin{flushleft}
        \small
        ФАКУЛЬТЕТ
        \underline{<<Информатика и системы управления>>\ \ \ \ \ \ \ 
        \ \ \ \ \ \ \ \ \ \ \ \ \ \ \ \ \ \ \ \ \ \ \ \ \ \ \ \ \ \ \ 
    \ \ \ \ \ \ \ \ \ \ \ \ \ \ \ } \\
        КАФЕДРА
        \underline{<<Программное обеспечение ЭВМ и
        информационные технологии>>
        \ \ \ \ \ \ \ \ \ \ \ \ \ \ \ \ \ \ \ \ }
    \end{flushleft}

    \vspace{4cm}

    \begin{center}
        \textbf{Лабораторная работа № 17} \\ 
        \hfill
        
        \textbf{Формирование эффективных программ на Prolog} \\
        \vspace{0.5cm}
        \textbf{} \\
    \end{center}

    \vspace{4cm}

    \begin{flushleft}
        \begin{tabular}{ll}
            \textbf{Дисциплина} & Функциональное и логическое программирование \\
            \textbf{Студент} & Сиденко А.Г. \\
            \textbf{Группа} & ИУ7-63Б \\
            \textbf{Преподаватель} & Толпинская Н.Б., Строганов Ю.В.  \\
        \end{tabular}
    \end{flushleft}

    \vspace{4cm}

   \begin{center}
        Москва, 2020 г.
    \end{center}

\end{titlepage}

\textbf{Задание}

\begin{enumerate}
\item Максимум из двух чисел
\begin{enumerate}
\item без использования отсечения,
\item с использованием отсечения;
\end{enumerate}
\item Максимум из трех чисел 
\begin{enumerate}
\item без использования отсечения,
\item с использованием отсечения;
\end{enumerate}
\end{enumerate}

Убедиться в правильности результатов.

Для каждого случая пункта 2 обосновать необходимость всех условий тела. 

Для одного из вариантов вопроса и каждого варианта задания 2 составить таблицу, отражающую конкретный порядок работы системы

\hfill

\textbf{Программа}

\begin{lstlisting}
predicates
  max2(integer, integer, integer).
  max2v(integer, integer, integer).
  max3(integer, integer, integer, integer).
  max3v(integer, integer, integer, integer).
clauses
  max2(A, B, A) :- A >= B.
  max2(A, B, B) :- B > A.
  
  max2v(A, B, A) :- A > B,!.
  max2v(_, B, B).
  
  % Максимум из 3 чисел
  % Первое утверждение, если первое число самое большое,  
  % или все равны, или равны первое и второе, или
  % первое и третье, и максимальны
  max3(A, B, C, A) :- A >= B, A >= C.
  % Второе утверждение, если второе число самое большое, или 
  % второе и третье равны
  max3(A, B, C, B) :- B > A, B >= C.
  % Второе утверждение, если третье число самое большое
  max3(A, B, C, C) :- C > A, C > B.
  % Знаки нестрогого неравенства присутствуют не во всех
  % утверждениях, чтобы не было повторений
  
  % Максимум из 3 чисел с использованием отсечения
  % Первое утверждение, если первое число строго самое большое
  max3v(A, B, C, A) :- A > B, A > C,!.
  % Второе утверждение, если второе число больше третьего,
  % может быть равно первому
  max3v(_, B, C, B) :- B > C,!.
  % Третье утверждение, все остальные случаи (последнее
  % максимальное, все равны и т.д.)
  max3v(_, _, C, C).
\end{lstlisting}

\textbf{Приведем таблицу для задания 2, пункт а. }
\begin{lstlisting}
goal
  max3(1, 5, 2, Rez).
\end{lstlisting}

\begin{longtable}{|p{0.5cm}|p{4cm}|p{7cm}|p{5.5cm}|}
	\hline
 	№ шага & Состояние резольвенты & Сравниваемые термы; результат; подстановка, если есть  & Дальнейшие действия: прямой ход или откат \\ \hline
	1 & max3(1, 5, 2, Rez) & По max3(1, 5, 2, Rez) ищется системой определение отношения (по имени предиката и списку (числу) аргументов) & Определение отношения найдено, заносится в стек max3(1, 5, 2, Rez), прямой ход \\ \hline
	2 &1>=5, 1>=2& Начинает <<раскрываться>> правило, т.е. доказывается каждое целевое утверждение в теле правила последовательно слева направо
	A>=B,A>=C
	
	& Прямой ход\\ \hline
	
	3 &1>=5, 1>=2& 1>=5	
	& Значение утверждения false, переход к следующему определению\\ \hline

	4 &5>1, 5>=2& Начинает <<раскрываться>> правило, т.е. доказывается каждое целевое утверждение в теле правила последовательно слева направо
	B>A,B>=C
	
	& Прямой ход\\ \hline
	5 &5>1, 5>=2& 5>1 & Значение утверждения true, переход к следующему целевому утверждению\\ \hline
	6 &5>1, 5>=2& 5>=2 & Значение утверждения true, переход к следующему целевому утверждению\\ \hline
	7 && & Резольента пуста, вывод результата, переход к следующему определению\\ \hline
	
	8 &2>1, 2>5& Начинает <<раскрываться>> правило, т.е. доказывается каждое целевое утверждение в теле правила последовательно слева направо
	C>A,C>B
	
	& Прямой ход\\ \hline
	
	9 &2>1, 2>5& 2>1	
	& Значение утверждения true, переход к следующему целевому утверждению\\ \hline
	
	10 &2>1, 2>5& 2>5	
	& Значение утверждения false, переход к следующему определению\\ \hline
         11&&&В базе знаний больше ни одного утверждения с заданным именем, возврат, достаем из стека max3(1, 5, 2, Rez)  \\ \hline
	12 & &  & Стек пуст, завершение программы \\ \hline

\end{longtable}

\hfill

\textbf{Приведем таблицу для задания 2, пункт b. }
\begin{lstlisting}
goal
  max3v(1, 5, 2, Rez).
\end{lstlisting}

\begin{longtable}{|p{0.5cm}|p{4cm}|p{7cm}|p{5.5cm}|}
	\hline
 	№ шага & Состояние резольвенты & Сравниваемые термы; результат; подстановка, если есть  & Дальнейшие действия: прямой ход или откат \\ \hline
	1 & max3v(1, 5, 2, Rez) & По max3v(1, 5, 2, Rez) ищется системой определение отношения (по имени предиката и списку (числу) аргументов) & Определение отношения найдено, заносится в стек max3v(1, 5, 2, Rez), прямой ход \\ \hline
	2 &1>5, 1>2& Начинает <<раскрываться>> правило, т.е. доказывается каждое целевое утверждение в теле правила последовательно слева направо
	A>B,A>C
	
	& Прямой ход\\ \hline
	
	3 &1>5, 1>2& 1>5	
	& Значение утверждения false, переход к следующему определению\\ \hline

	4 &5>2& Начинает <<раскрываться>> правило, т.е. доказывается каждое целевое утверждение в теле правила последовательно слева направо
	B>C
	
	& Прямой ход\\ \hline
	5 &5>2& 5>2 & Значение утверждения true, переход к следующему целевому утверждению\\ \hline
	6 && & Резольента пуста, вывод результата, отсечение\\ \hline

         7&&&Возврат, достаем из стека max3v(1, 5, 2, Rez)  \\ \hline
	8 & &  & Стек пуст, завершение программы \\ \hline

\end{longtable}

\hfill

\textbf{Вывод}

Где это возможно необходимо использовать <<!>> -- отсечение. Которое сократит количество выполняемых унификаций для достижения максимальной эффективности работы системы. 

\hfill

\textbf{Ответы на вопросы}

\begin{enumerate} 

\item Какое первое состояние резольвенты?

Вопрос. 

\item В каком случае система запускает алгоритм унификации? (Как эту необходимость на формальном уровне распознает система?)

Пролог выполняет унификацию в двух случаях: когда цель сопоставляется с заголовком предложения или когда используется знак равенства, который является инфиксным предикатом (предикатом, который расположен между своими аргументами, а не перед ними).

\item Каково назначение использования алгоритма унификации? 

Унификация двух термов -- это основной шаг доказательства. В процессе работы система выполняет большое число унификаций.
\textbf{Унификация} -- операция, которая позволяет формализовать процесс логического вывода. 

Унификация представляет собой процесс сопоставления цели с фактами и правилами базы знаний. Цель может быть согласована, если она может быть сопоставлена с заголовком какого-либо предложения базы.

\item Каков результат работы алгоритма унификации? 

Результатом его работы является последняя из построенных подстановок. 

\item В каких пределах программы уникальны переменные? 

Областью действия переменной в Прологе является одно предложение. В разных предложениях может использоваться одно имя переменной для обозначения разных объектов. Исключением является анонимная переменная. Каждая анонимная переменная -- это отдельный объект.

\item Как применяется подстановка, полученная с помощью алгоритма унификации?

При согласовании переменные получают значения, указанные с другой стороны от знака <<=>>, если переменные еще не были связаны. Переменные становятся связанными и после успешного согласования всех целевых утверждений, будет напечатано значение связанных переменных.

\item Как изменяется резольвента?

Резольвента - текущая цель, существующая на любой стадии вычислений. Резольвенты порождаются целью и каким-либо правилом или фактом, которые просматриваются последовательно сверху вниз. Если резольвента существует при наиболее общей унификации, она вычисляется. Если пустая резольвента с помощью такой стратегии не найдена, то ответ на вопрос отрицателен.

\item В каких случаях запускается механизм отката?

Откат дает возможность получить много решений в одном вопросе к программе. 

Во всех точках программы, где существуют альтернативы, в стек заносятся точки возврата. 

Если впоследствии окажется, что выбранный вариант не приводит к успеху, то осуществляется откат к последней из имеющихся в стеке точек программы, где был выбран один из альтернативных вариантов. 

Выбирается очередной вариант, программа продолжает свою работу. Если все варианты в точке уже были использованы, то регистрируется неудачное завершение и осуществляется переход на предыдущую точку возврата, если такая есть. 

При откате все связанные переменные, которые были означены после этой точки, опять освобождаются.

\end{enumerate}
 
\end{document}