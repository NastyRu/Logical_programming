\documentclass[a4paper,14pt]{extreport} % формат документа

\usepackage{amsmath}
\usepackage{cmap} % поиск в ПДФ
\usepackage[T2A]{fontenc} % кодировка
\usepackage[utf8]{inputenc} % кодировка исходного текста
\usepackage[english,russian]{babel} % локализация и переносы
\usepackage[left = 2cm, right = 1cm, top = 2cm, bottom = 2 cm]{geometry} % поля
\usepackage{listings}
\usepackage{graphicx} % для вставки рисунков
\usepackage{amsmath}
\usepackage{float}
\usepackage{longtable}
\usepackage{multirow}
\graphicspath{{pictures/}}
\DeclareGraphicsExtensions{.pdf,.png,.jpg}
\newcommand{\anonsection}[1]{\section*{#1}\addcontentsline{toc}{section}{#1}}

\lstset{ %
	language=Prolog,                % Язык программирования 
	numbers=left,                   % С какой стороны нумеровать          
	frame=single,                    % Добавить рамку
	 escapebegin=\begin{russian}\commentfont,
    escapeend=\end{russian},
    literate={Ö}{{\"O}}1
    {Ä}{{\"A}}1
    {Ü}{{\"U}}1
    {ß}{{\ss}}1
    {ü}{{\"u}}1
    {ä}{{\"a}}1
    {ö}{{\"o}}1
    {~}{{\textasciitilde}}1
    {а}{{\selectfont\char224}}1
    {б}{{\selectfont\char225}}1
    {в}{{\selectfont\char226}}1
    {г}{{\selectfont\char227}}1
    {д}{{\selectfont\char228}}1
    {е}{{\selectfont\char229}}1
    {ё}{{\"e}}1
    {ж}{{\selectfont\char230}}1
    {з}{{\selectfont\char231}}1
    {и}{{\selectfont\char232}}1
    {й}{{\selectfont\char233}}1
    {к}{{\selectfont\char234}}1
    {л}{{\selectfont\char235}}1
    {м}{{\selectfont\char236}}1
    {н}{{\selectfont\char237}}1
    {о}{{\selectfont\char238}}1
    {п}{{\selectfont\char239}}1
    {р}{{\selectfont\char240}}1
    {с}{{\selectfont\char241}}1
    {т}{{\selectfont\char242}}1
    {у}{{\selectfont\char243}}1
    {ф}{{\selectfont\char244}}1
    {х}{{\selectfont\char245}}1
    {ц}{{\selectfont\char246}}1
    {ч}{{\selectfont\char247}}1
    {ш}{{\selectfont\char248}}1
    {щ}{{\selectfont\char249}}1
    {ъ}{{\selectfont\char250}}1
    {ы}{{\selectfont\char251}}1
    {ь}{{\selectfont\char252}}1
    {э}{{\selectfont\char253}}1
    {ю}{{\selectfont\char254}}1
    {я}{{\selectfont\char255}}1
    {А}{{\selectfont\char192}}1
    {Б}{{\selectfont\char193}}1
    {В}{{\selectfont\char194}}1
    {Г}{{\selectfont\char195}}1
    {Д}{{\selectfont\char196}}1
    {Е}{{\selectfont\char197}}1
    {Ё}{{\"E}}1
    {Ж}{{\selectfont\char198}}1
    {З}{{\selectfont\char199}}1
    {И}{{\selectfont\char200}}1
    {Й}{{\selectfont\char201}}1
    {К}{{\selectfont\char202}}1
    {Л}{{\selectfont\char203}}1
    {М}{{\selectfont\char204}}1
    {Н}{{\selectfont\char205}}1
    {О}{{\selectfont\char206}}1
    {П}{{\selectfont\char207}}1
    {Р}{{\selectfont\char208}}1
    {С}{{\selectfont\char209}}1
    {Т}{{\selectfont\char210}}1
    {У}{{\selectfont\char211}}1
    {Ф}{{\selectfont\char212}}1
    {Х}{{\selectfont\char213}}1
    {Ц}{{\selectfont\char214}}1
    {Ч}{{\selectfont\char215}}1
    {Ш}{{\selectfont\char216}}1
    {Щ}{{\selectfont\char217}}1
    {Ъ}{{\selectfont\char218}}1
    {Ы}{{\selectfont\car219}}1
    {Ь}{{\selectfont\char220}}1
    {Э}{{\selectfont\char221}}1
    {Ю}{{\selectfont\char222}}1
    {Я}{{\selectfont\char223}}1
    {і}{{\selectfont\char105}}1
    {ї}{{\selectfont\char168}}1
    {є}{{\selectfont\char185}}1
    {ґ}{{\selectfont\char160}}1
    {І}{{\selectfont\char73}}1
    {Ї}{{\selectfont\char136}}1
    {Є}{{\selectfont\char153}}1
    {Ґ}{{\selectfont\char128}}1
}

\begin{document}
\begin{titlepage}

    \begin{table}[H]
        \centering
        \footnotesize
        \begin{tabular}{cc}
            \multirow{8}{*}{\includegraphics[scale=0.35]{bmstu.jpg}}
            & \\
            & \\
            & \textbf{Министерство науки и высшего образования Российской Федерации} \\
            & \textbf{Федеральное государственное бюджетное образовательное учреждение} \\
            & \textbf{высшего образования} \\
            & \textbf{<<Московский государственный технический} \\
            & \textbf{университет имени Н.Э. Баумана>>} \\
            & \textbf{(МГТУ им. Н.Э. Баумана)} \\
        \end{tabular}
    \end{table}

    \vspace{-2.5cm}

    \begin{flushleft}
        \rule[-1cm]{\textwidth}{3pt}
        \rule{\textwidth}{1pt}
    \end{flushleft}

    \begin{flushleft}
        \small
        ФАКУЛЬТЕТ
        \underline{<<Информатика и системы управления>>\ \ \ \ \ \ \ 
        \ \ \ \ \ \ \ \ \ \ \ \ \ \ \ \ \ \ \ \ \ \ \ \ \ \ \ \ \ \ \ 
    \ \ \ \ \ \ \ \ \ \ \ \ \ \ \ } \\
        КАФЕДРА
        \underline{<<Программное обеспечение ЭВМ и
        информационные технологии>>
        \ \ \ \ \ \ \ \ \ \ \ \ \ \ \ \ \ \ \ \ }
    \end{flushleft}

    \vspace{4cm}

    \begin{center}
        \textbf{Лабораторная работа № 18} \\ 
        \hfill
        
        \textbf{Формирование эффективных программ на Prolog} \\
        \vspace{0.5cm}
        \textbf{} \\
    \end{center}

    \vspace{4cm}

    \begin{flushleft}
        \begin{tabular}{ll}
            \textbf{Дисциплина} & Функциональное и логическое программирование \\
            \textbf{Студент} & Сиденко А.Г. \\
            \textbf{Группа} & ИУ7-63Б \\
            \textbf{Преподаватель} & Толпинская Н.Б., Строганов Ю.В.  \\
        \end{tabular}
    \end{flushleft}

    \vspace{4cm}

   \begin{center}
        Москва, 2020 г.
    \end{center}

\end{titlepage}

\textbf{Задание}

Используя хвостовую рекурсию, разработать программу, позволяющую найти 

\begin{enumerate}
\item n!
\item n-е число Фибоначчи
\end{enumerate}

Убедиться в правильности результатов.

Для одного из вариантов вопроса и каждого задания составить таблицу, отражающую конкретный порядок работы системы

\hfill

\textbf{Программа}

\begin{lstlisting}
predicates
  factorial(integer, integer).
  fibonachi(integer, integer).
clauses
  factorial(0, 1):- !.
  factorial(Next, FactorialNext):- 
    Cur = Next - 1, 
    factorial(Cur, FactorialCur),
    FactorialNext = FactorialCur * Next.
    
  fibonachi(0, 0):- !.
  fibonachi(1, 1):- !.
  fibonachi(Next, FibonachiNext):- 
    Cur = Next - 1,
    Prev = Cur - 1,
    fibonachi(Cur, FibonachiCur),
    fibonachi(Prev, FibonachiPrev),
    FibonachiNext = FibonachiCur + FibonachiPrev.
\end{lstlisting}

\textbf{Приведем таблицу для задания 1. }
\begin{lstlisting}
goal
  factorial(5, Rez).
\end{lstlisting}

\begin{longtable}{|p{0.5cm}|p{5cm}|p{6cm}|p{5.5cm}|}
	\hline
 	№ шага & Состояние резольвенты & Сравниваемые термы; результат; подстановка, если есть  & Дальнейшие действия: прямой ход или откат \\ \hline
	1 & factorial(5, Rez) & По factorial(5, Rez) ищется системой определение отношения (по имени предиката и списку (числу) аргументов) & Определение отношения найдено, заносится в стек factorial(5, Rez), прямой ход \\ \hline
	2 &Cur = 5 - 1, factorial(Cur, FactorialCur), FactorialNext = FactorialCur * 5 & Начинает <<раскрываться>> правило, т.е. доказывается каждое целевое утверждение в теле правила последовательно слева направо
	Cur = Next - 1, factorial(Cur, FactorialCur), FactorialNext = FactorialCur * Next & Прямой ход\\ \hline
	
	3 & Cur = 5 - 1, factorial(Cur, FactorialCur), FactorialNext = FactorialCur * 5 & Cur = 5 - 1	
	& Значение утверждения true, Cur = 4, переход к следующему определению\\ \hline

	4 & factorial(4, FactorialCur), FactorialNext = FactorialCur * 5 & По factorial(4, FactorialCur) ищется системой определение отношения (по имени предиката и списку (числу) аргументов) & Определение отношения найдено, заносится в стек factorial(4, FactorialCur), прямой ход \\ \hline
	
	5 & Cur = 4 - 1, factorial(Cur, FactorialCur), FactorialNext = FactorialCur * 4, FactorialNext = FactorialCur * 5 & Cur = 4 - 1	
	& Значение утверждения true, Cur = 3, переход к следующему определению\\ \hline
	
	6 & factorial(3, FactorialCur), FactorialNext = FactorialCur * 4, FactorialNext = FactorialCur * 5 & По factorial(3, FactorialCur) ищется системой определение отношения (по имени предиката и списку (числу) аргументов) & Определение отношения найдено, заносится в стек factorial(3, FactorialCur), прямой ход \\ \hline
	
	7 & Cur = 3 - 1, factorial(Cur, FactorialCur), FactorialNext = FactorialCur * 3, FactorialNext = FactorialCur * 4, FactorialNext = FactorialCur * 5 & Cur = 4 - 1	
	& Значение утверждения true, Cur = 2, переход к следующему определению\\ \hline
	
	8 & factorial(2, FactorialCur), FactorialNext = FactorialCur * 3, FactorialNext = FactorialCur * 4, FactorialNext = FactorialCur * 5 & По factorial(2, FactorialCur) ищется системой определение отношения (по имени предиката и списку (числу) аргументов) & Определение отношения найдено, заносится в стек factorial(2, FactorialCur), прямой ход \\ \hline
	
	9 & Cur = 2 - 1, factorial(Cur, FactorialCur), FactorialNext = FactorialCur * 2, FactorialNext = FactorialCur * 3, FactorialNext = FactorialCur * 4, FactorialNext = FactorialCur * 5 & Cur = 4 - 1	
	& Значение утверждения true, Cur = 1, переход к следующему определению\\ \hline
	
	10 & factorial(1, FactorialCur), FactorialNext = FactorialCur * 2, FactorialNext = FactorialCur * 3, FactorialNext = FactorialCur * 4, FactorialNext = FactorialCur * 5 & По factorial(1, FactorialCur) ищется системой определение отношения (по имени предиката и списку (числу) аргументов) & Определение отношения найдено, заносится в стек factorial(1, FactorialCur), прямой ход \\ \hline
	
	11 &Cur = 1 - 1, factorial(Cur, FactorialCur), FactorialNext = FactorialCur * 1, FactorialNext = FactorialCur * 2, FactorialNext = FactorialCur * 3, FactorialNext = FactorialCur * 4, FactorialNext = FactorialCur * 5 & Cur = 4 - 1	
	& Значение утверждения true, Cur = 0, переход к следующему определению\\ \hline
	
	12 & factorial(0, FactorialCur), FactorialNext = FactorialCur * 1, FactorialNext = FactorialCur * 2, FactorialNext = FactorialCur * 3, FactorialNext = FactorialCur * 4, FactorialNext = FactorialCur * 5 & По factorial(0, FactorialCur) ищется системой определение отношения (по имени предиката и списку (числу) аргументов) & Определение отношения найдено, заносится в стек factorial(0, FactorialCur), прямой ход \\ \hline
	
	13 & FactorialNext = 1 * 1, FactorialNext = FactorialCur * 2, FactorialNext = FactorialCur * 3, FactorialNext = FactorialCur * 4, FactorialNext = FactorialCur * 5 & Унификация factorial(0, FactorialCur) и factorial(0, 1) & FactorialCur = 1, извлечение из стека factorial(0, FactorialCur) \\ \hline
	
	14 & FactorialNext = 1 * 2, FactorialNext = FactorialCur * 3, FactorialNext = FactorialCur * 4, FactorialNext = FactorialCur * 5 &  FactorialNext = 1& Извлечение из стека factorial(1, FactorialCur) \\ \hline
	
	15 & FactorialNext = 2 * 3, FactorialNext = FactorialCur * 4, FactorialNext = FactorialCur * 5 &  FactorialNext = 2& Извлечение из стека factorial(2, FactorialCur) \\ \hline
	
	16 & FactorialNext = 6 * 4, FactorialNext = FactorialCur * 5 &  FactorialNext = 6& Извлечение из стека factorial(3, FactorialCur) \\ \hline
	
	17 & FactorialNext = 24 * 5 &  FactorialNext = 24& Извлечение из стека factorial(4, FactorialCur) \\ \hline
	
	18 &  &  FactorialNext = 120& Резольвента пуста, вывод результата \\ \hline
	
	19 &  &  & Стек пуст, завершение программы \\ \hline

\end{longtable}

\hfill

\textbf{Приведем таблицу для задания 2. }
\begin{lstlisting}
goal
  fibonachi(4, Rez).
\end{lstlisting}

\begin{longtable}{|p{0.5cm}|p{5cm}|p{6cm}|p{5.5cm}|}
	\hline
 	№ шага & Состояние резольвенты & Сравниваемые термы; результат; подстановка, если есть  & Дальнейшие действия: прямой ход или откат \\ \hline
	1 & fibonachi(4, Rez) & По fibonachi(4, Rez) ищется системой определение отношения (по имени предиката и списку (числу) аргументов) & Определение отношения найдено, заносится в стек fibonachi(4, Rez), прямой ход \\ \hline
	2 &Cur = 4 - 1, Prev = Cur - 1, fibonachi(Cur, FibonachiCur), fibonachi(Prev, FibonachiPrev), FibonachiNext = FibonachiCur + FibonachiPrev & Начинает <<раскрываться>> правило, т.е. доказывается каждое целевое утверждение в теле правила последовательно слева направо
	Cur = Next - 1, Prev = Cur - 1, fibonachi(Cur, FibonachiCur), fibonachi(Prev, FibonachiPrev), FibonachiNext = FibonachiCur + FibonachiPrev & Прямой ход\\ \hline
	
	3 & Prev = 3 - 1, fibonachi(3, FibonachiCur), fibonachi(Prev, FibonachiPrev), FibonachiNext = FibonachiCur + FibonachiPrev & Cur = 4 - 1	
	& Значение утверждения true, Cur = 3, переход к следующему определению\\ \hline
	
	4 & fibonachi(3, FibonachiCur), fibonachi(2, FibonachiPrev), FibonachiNext = FibonachiCur + FibonachiPrev & Prev = 3 - 1	
	& Значение утверждения true, Prev = 2, переход к следующему определению\\ \hline
	
	5 & fibonachi(3, FibonachiCur), fibonachi(2, FibonachiPrev), FibonachiNext = FibonachiCur + FibonachiPrev &  По fibonachi(3, FibonachiCur) ищется системой определение отношения (по имени предиката и списку (числу) аргументов) & Определение отношения найдено, заносится в стек fibonachi(3, FibonachiCur), прямой ход \\ \hline
	
	6 & Cur = 3 - 1, Prev = Cur - 1, fibonachi(Cur, FibonachiCur), fibonachi(Prev, FibonachiPrev), FibonachiNext = FibonachiCur + FibonachiPrev, fibonachi(2, FibonachiPrev), FibonachiNext = FibonachiCur + FibonachiPrev &  Начинает <<раскрываться>> правило, т.е. доказывается каждое целевое утверждение в теле правила последовательно слева направо
	Cur = Next - 1, Prev = Cur - 1, fibonachi(Cur, FibonachiCur), fibonachi(Prev, FibonachiPrev), FibonachiNext = FibonachiCur + FibonachiPrev & Прямой ход\\ \hline
	
	7 & Prev = 2 - 1, fibonachi(2, FibonachiCur), fibonachi(Prev, FibonachiPrev), FibonachiNext = FibonachiCur + FibonachiPrev, fibonachi(2, FibonachiPrev), FibonachiNext = FibonachiCur + FibonachiPrev  & Cur = 3 - 1	
	& Значение утверждения true, Cur = 2, переход к следующему определению\\ \hline
	
	8 & fibonachi(2, FibonachiCur), fibonachi(1, FibonachiPrev), FibonachiNext = FibonachiCur + FibonachiPrev, fibonachi(2, FibonachiPrev), FibonachiNext = FibonachiCur + FibonachiPrev  & Prev = 2 - 1	
	& Значение утверждения true, Prev = 1, переход к следующему определению\\ \hline
	
	9 & fibonachi(2, FibonachiCur), fibonachi(1, FibonachiPrev), FibonachiNext = FibonachiCur + FibonachiPrev, fibonachi(2, FibonachiPrev), FibonachiNext = FibonachiCur + FibonachiPrev  & По fibonachi(2, FibonachiCur) ищется системой определение отношения (по имени предиката и списку (числу) аргументов) & Определение отношения найдено, заносится в стек fibonachi(2, FibonachiCur), прямой ход \\ \hline
	
	10 & Cur = 2 - 1, Prev = Cur - 1, fibonachi(Cur, FibonachiCur), fibonachi(Prev, FibonachiPrev), FibonachiNext = FibonachiCur + FibonachiPrev, fibonachi(1, FibonachiPrev), FibonachiNext = FibonachiCur + FibonachiPrev, fibonachi(2, FibonachiPrev), FibonachiNext = FibonachiCur + FibonachiPrev &  Начинает <<раскрываться>> правило, т.е. доказывается каждое целевое утверждение в теле правила последовательно слева направо
	Cur = Next - 1, Prev = Cur - 1, fibonachi(Cur, FibonachiCur), fibonachi(Prev, FibonachiPrev), FibonachiNext = FibonachiCur + FibonachiPrev & Прямой ход\\ \hline
	
	11 & Prev = 1 - 1, fibonachi(1, FibonachiCur), fibonachi(Prev, FibonachiPrev), FibonachiNext = FibonachiCur + FibonachiPrev, fibonachi(1, FibonachiPrev), FibonachiNext = FibonachiCur + FibonachiPrev, fibonachi(2, FibonachiPrev), FibonachiNext = FibonachiCur + FibonachiPrev   & Cur = 2 - 1	
	& Значение утверждения true, Cur = 1, переход к следующему определению\\ \hline
	
	12 & fibonachi(1, FibonachiCur), fibonachi(0, FibonachiPrev), FibonachiNext = FibonachiCur + FibonachiPrev, fibonachi(1, FibonachiPrev), FibonachiNext = FibonachiCur + FibonachiPrev, fibonachi(2, FibonachiPrev), FibonachiNext = FibonachiCur + FibonachiPrev   & Prev = 1 - 1	
	& Значение утверждения true, Prev = 0, переход к следующему определению\\ \hline
	
	13 & fibonachi(1, FibonachiCur), fibonachi(0, FibonachiPrev), FibonachiNext = FibonachiCur + FibonachiPrev, fibonachi(1, FibonachiPrev), FibonachiNext = FibonachiCur + FibonachiPrev, fibonachi(2, FibonachiPrev), FibonachiNext = FibonachiCur + FibonachiPrev  & По fibonachi(1, FibonachiCur) ищется системой определение отношения (по имени предиката и списку (числу) аргументов) & Определение отношения найдено, заносится в стек fibonachi(1, FibonachiCur), прямой ход \\ \hline
	
	14 & fibonachi(0, FibonachiPrev), FibonachiNext = FibonachiCur + FibonachiPrev, fibonachi(1, FibonachiPrev), FibonachiNext = FibonachiCur + FibonachiPrev, fibonachi(2, FibonachiPrev), FibonachiNext = FibonachiCur + FibonachiPrev  & Унификация fibonachi(1, FibonachiCur) и fibonachi(1, 1) & FibonachiCur = 1, извлекается из стека fibonachi(1, FibonachiCur), прямой ход \\ \hline
	
	15 & fibonachi(0, FibonachiPrev), FibonachiNext = 1 + FibonachiPrev, fibonachi(1, FibonachiPrev), FibonachiNext = FibonachiCur + FibonachiPrev, fibonachi(2, FibonachiPrev), FibonachiNext = FibonachiCur + FibonachiPrev  & По fibonachi(0, FibonachiCur) ищется системой определение отношения (по имени предиката и списку (числу) аргументов) & Определение отношения найдено, заносится в стек fibonachi(0, FibonachiCur), прямой ход \\ \hline
	
	16 & FibonachiNext = 1 + FibonachiPrev, fibonachi(1, FibonachiPrev), FibonachiNext = FibonachiCur + FibonachiPrev, fibonachi(2, FibonachiPrev), FibonachiNext = FibonachiCur + FibonachiPrev  & Унификация fibonachi(0, FibonachiCur) и fibonachi(0, 0) & FibonachiPrev = 0, извлекается из стека fibonachi(0, FibonachiCur), прямой ход \\ \hline
	
	17 & FibonachiNext = 1 + 0, fibonachi(1, FibonachiPrev), FibonachiNext = FibonachiCur + FibonachiPrev, fibonachi(2, FibonachiPrev), FibonachiNext = FibonachiCur + FibonachiPrev  & FibonachiNext = 1 + 0 & FibonachiCur = 1, прямой ход \\ \hline
	
	18 & fibonachi(1, FibonachiPrev), FibonachiNext = 1 + FibonachiPrev, fibonachi(2, FibonachiPrev), FibonachiNext = FibonachiCur + FibonachiPrev  & По fibonachi(1, FibonachiPrev) ищется системой определение отношения (по имени предиката и списку (числу) аргументов) & Определение отношения найдено, заносится в стек fibonachi(1, FibonachiPrev), прямой ход \\ \hline
	
	19 & FibonachiNext = 1 + FibonachiPrev, fibonachi(2, FibonachiPrev), FibonachiNext = FibonachiCur + FibonachiPrev  & Унификация fibonachi(1, FibonachiPrev) и fibonachi(1, 1) & FibonachiPrev = 1, извлекается из стека fibonachi(1, FibonachiCur), прямой ход \\ \hline
	
	20 & FibonachiNext = 1 + 1, fibonachi(2, FibonachiPrev), FibonachiNext = FibonachiCur + FibonachiPrev  & FibonachiNext = 1 + 1 & FibonachiCur = 2, прямой ход \\ \hline
	
	21 & fibonachi(2, FibonachiPrev), FibonachiNext = 2 + FibonachiPrev  & По fibonachi(2, FibonachiPrev) ищется системой определение отношения (по имени предиката и списку (числу) аргументов) & Определение отношения найдено, заносится в стек fibonachi(2, FibonachiPrev), прямой ход \\ \hline	
	
	22 & Cur = 2 - 1, Prev = Cur - 1, fibonachi(Cur, FibonachiCur), fibonachi(Prev, FibonachiPrev), FibonachiNext = FibonachiCur + FibonachiPrev, FibonachiNext = 2 + FibonachiPrev &  Начинает <<раскрываться>> правило, т.е. доказывается каждое целевое утверждение в теле правила последовательно слева направо
	Cur = Next - 1, Prev = Cur - 1, fibonachi(Cur, FibonachiCur), fibonachi(Prev, FibonachiPrev), FibonachiNext = FibonachiCur + FibonachiPrev & Прямой ход\\ \hline
	
	23 & Prev = 1 - 1, fibonachi(1, FibonachiCur), fibonachi(Prev, FibonachiPrev), FibonachiNext = FibonachiCur + FibonachiPrev, FibonachiNext = 2 + FibonachiPrev    & Cur = 2 - 1	
	& Значение утверждения true, Cur = 1, переход к следующему определению\\ \hline
	
	24 & fibonachi(1, FibonachiCur), fibonachi(0, FibonachiPrev), FibonachiNext = FibonachiCur + FibonachiPrev, FibonachiNext = 2 + FibonachiPrev   & Prev = 1 - 1	
	& Значение утверждения true, Prev = 0, переход к следующему определению\\ \hline
	
	25 & fibonachi(1, FibonachiCur), fibonachi(0, FibonachiPrev), FibonachiNext = FibonachiCur + FibonachiPrev, FibonachiNext = 2 + FibonachiPrev  & По fibonachi(1, FibonachiCur) ищется системой определение отношения (по имени предиката и списку (числу) аргументов) & Определение отношения найдено, заносится в стек fibonachi(1, FibonachiCur), прямой ход \\ \hline
	
	26 & fibonachi(0, FibonachiPrev), FibonachiNext = FibonachiCur + FibonachiPrev, FibonachiNext = 2 + FibonachiPrev  & Унификация fibonachi(1, FibonachiCur) и fibonachi(1, 1) & FibonachiCur = 1, извлекается из стека fibonachi(1, FibonachiCur), прямой ход \\ \hline
	
	27 & fibonachi(0, FibonachiPrev), FibonachiNext = 1 + FibonachiPrev, FibonachiNext = 2 + FibonachiPrev  & По fibonachi(0, FibonachiCur) ищется системой определение отношения (по имени предиката и списку (числу) аргументов) & Определение отношения найдено, заносится в стек fibonachi(0, FibonachiCur), прямой ход \\ \hline

	28 & FibonachiNext = 1 + FibonachiPrev, FibonachiNext = 2 + FibonachiPrev  & Унификация fibonachi(0, FibonachiCur) и fibonachi(0, 0) & FibonachiPrev = 0, извлекается из стека fibonachi(0, FibonachiCur), прямой ход \\ \hline
	
	29 & FibonachiNext = 1 + 0, FibonachiNext = 2 + FibonachiPrev  & FibonachiNext = 1 + 0 & FibonachiPrev = 1, извлекается из стека fibonachi(3, FibonachiCur), прямой ход \\ \hline
	
	30 &  & FibonachiNext = 2 + 1  & FibonachiNext = 3, извлекается из стека fibonachi(4, FibonachiCur), Резольвента пуста, вывод результата \\ \hline
	
	31 &  &  & Стек пуст, завершение программы \\ \hline

\end{longtable}

\hfill

\textbf{Вывод}

Эффективный способ организации рекурсии --  хвостовая рекурсия. Эффективность рекурсивной процедуры повышается благодаря отсечению неперспективных путей поиска решения. Используя <<!>> -- отсечение. Которое сократит количество выполняемых унификаций для достижения максимальной эффективности работы системы. 

\textbf{Ответы на вопросы}

\begin{enumerate} 

\item Что такое рекурсия? Как организуется хвостовая рекурсия в Prolog? Как организовать выход из рекурсии в Prolog?

Рекурсия позволяет использовать в процессе определения предиката его самого. 

Хвостовая рекурсия: Для ее осуществления рекурсивный вызов определяемого предиката должен быть последней подцелью в теле рекурсивного правила и к моменту рекурсивного вызова не должно остаться точек возврата (непроверенных альтернатив). 

Параметры должны изменяться на каждом шаге так, чтобы в итоге либо сработал базис рекурсии, либо условие выхода из рекурсии, размещенное в самом правиле.

\item Какое первое состояние резольвенты?

Вопрос. 

\item В каком случае система запускает алгоритм унификации? Каково назначение использования алгоритма унификации? Каков результат работы алгоритма унификации? 

Пролог выполняет унификацию в двух случаях: когда цель сопоставляется с заголовком предложения или когда используется знак равенства, который является инфиксным предикатом (предикатом, который расположен между своими аргументами, а не перед ними).

Унификация двух термов -- это основной шаг доказательства. В процессе работы система выполняет большое число унификаций.
\textbf{Унификация} -- операция, которая позволяет формализовать процесс логического вывода. 

Унификация представляет собой процесс сопоставления цели с фактами и правилами базы знаний. Цель может быть согласована, если она может быть сопоставлена с заголовком какого-либо предложения базы.

Результатом его работы является последняя из построенных подстановок. 

\item В каких пределах программы уникальны переменные? 

Областью действия переменной в Прологе является одно предложение. В разных предложениях может использоваться одно имя переменной для обозначения разных объектов. Исключением является анонимная переменная. Каждая анонимная переменная -- это отдельный объект.

\item Как применяется подстановка, полученная с помощью алгоритма унификации?

При согласовании переменные получают значения, указанные с другой стороны от знака <<=>>, если переменные еще не были связаны. Переменные становятся связанными и после успешного согласования всех целевых утверждений, будет напечатано значение связанных переменных.

\item Как изменяется резольвента?

Резольвента - текущая цель, существующая на любой стадии вычислений. Резольвенты порождаются целью и каким-либо правилом или фактом, которые просматриваются последовательно сверху вниз. Если резольвента существует при наиболее общей унификации, она вычисляется. Если пустая резольвента с помощью такой стратегии не найдена, то ответ на вопрос отрицателен.

\item В каких случаях запускается механизм отката?

Откат дает возможность получить много решений в одном вопросе к программе. 

Во всех точках программы, где существуют альтернативы, в стек заносятся точки возврата. 

Если впоследствии окажется, что выбранный вариант не приводит к успеху, то осуществляется откат к последней из имеющихся в стеке точек программы, где был выбран один из альтернативных вариантов. 

Выбирается очередной вариант, программа продолжает свою работу. Если все варианты в точке уже были использованы, то регистрируется неудачное завершение и осуществляется переход на предыдущую точку возврата, если такая есть. 

При откате все связанные переменные, которые были означены после этой точки, опять освобождаются.

\end{enumerate}
 
\end{document}