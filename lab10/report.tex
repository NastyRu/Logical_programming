\documentclass[a4paper,14pt]{extreport} % формат документа

\usepackage{amsmath}
\usepackage{cmap} % поиск в ПДФ
\usepackage[T2A]{fontenc} % кодировка
\usepackage[utf8]{inputenc} % кодировка исходного текста
\usepackage[english,russian]{babel} % локализация и переносы
\usepackage[left = 2cm, right = 1cm, top = 2cm, bottom = 2 cm]{geometry} % поля
\usepackage{listings}
\usepackage{graphicx} % для вставки рисунков
\usepackage{amsmath}
\usepackage{float}
\usepackage{longtable}
\usepackage{multirow}
\graphicspath{{pictures/}}
\DeclareGraphicsExtensions{.pdf,.png,.jpg}
\newcommand{\anonsection}[1]{\section*{#1}\addcontentsline{toc}{section}{#1}}

\lstset{ %
	language=Prolog,                % Язык программирования 
	numbers=left,                   % С какой стороны нумеровать          
	frame=single,                    % Добавить рамку
	 escapebegin=\begin{russian}\commentfont,
    escapeend=\end{russian},
    literate={Ö}{{\"O}}1
    {Ä}{{\"A}}1
    {Ü}{{\"U}}1
    {ß}{{\ss}}1
    {ü}{{\"u}}1
    {ä}{{\"a}}1
    {ö}{{\"o}}1
    {~}{{\textasciitilde}}1
    {а}{{\selectfont\char224}}1
    {б}{{\selectfont\char225}}1
    {в}{{\selectfont\char226}}1
    {г}{{\selectfont\char227}}1
    {д}{{\selectfont\char228}}1
    {е}{{\selectfont\char229}}1
    {ё}{{\"e}}1
    {ж}{{\selectfont\char230}}1
    {з}{{\selectfont\char231}}1
    {и}{{\selectfont\char232}}1
    {й}{{\selectfont\char233}}1
    {к}{{\selectfont\char234}}1
    {л}{{\selectfont\char235}}1
    {м}{{\selectfont\char236}}1
    {н}{{\selectfont\char237}}1
    {о}{{\selectfont\char238}}1
    {п}{{\selectfont\char239}}1
    {р}{{\selectfont\char240}}1
    {с}{{\selectfont\char241}}1
    {т}{{\selectfont\char242}}1
    {у}{{\selectfont\char243}}1
    {ф}{{\selectfont\char244}}1
    {х}{{\selectfont\char245}}1
    {ц}{{\selectfont\char246}}1
    {ч}{{\selectfont\char247}}1
    {ш}{{\selectfont\char248}}1
    {щ}{{\selectfont\char249}}1
    {ъ}{{\selectfont\char250}}1
    {ы}{{\selectfont\char251}}1
    {ь}{{\selectfont\char252}}1
    {э}{{\selectfont\char253}}1
    {ю}{{\selectfont\char254}}1
    {я}{{\selectfont\char255}}1
    {А}{{\selectfont\char192}}1
    {Б}{{\selectfont\char193}}1
    {В}{{\selectfont\char194}}1
    {Г}{{\selectfont\char195}}1
    {Д}{{\selectfont\char196}}1
    {Е}{{\selectfont\char197}}1
    {Ё}{{\"E}}1
    {Ж}{{\selectfont\char198}}1
    {З}{{\selectfont\char199}}1
    {И}{{\selectfont\char200}}1
    {Й}{{\selectfont\char201}}1
    {К}{{\selectfont\char202}}1
    {Л}{{\selectfont\char203}}1
    {М}{{\selectfont\char204}}1
    {Н}{{\selectfont\char205}}1
    {О}{{\selectfont\char206}}1
    {П}{{\selectfont\char207}}1
    {Р}{{\selectfont\char208}}1
    {С}{{\selectfont\char209}}1
    {Т}{{\selectfont\char210}}1
    {У}{{\selectfont\char211}}1
    {Ф}{{\selectfont\char212}}1
    {Х}{{\selectfont\char213}}1
    {Ц}{{\selectfont\char214}}1
    {Ч}{{\selectfont\char215}}1
    {Ш}{{\selectfont\char216}}1
    {Щ}{{\selectfont\char217}}1
    {Ъ}{{\selectfont\char218}}1
    {Ы}{{\selectfont\car219}}1
    {Ь}{{\selectfont\char220}}1
    {Э}{{\selectfont\char221}}1
    {Ю}{{\selectfont\char222}}1
    {Я}{{\selectfont\char223}}1
    {і}{{\selectfont\char105}}1
    {ї}{{\selectfont\char168}}1
    {є}{{\selectfont\char185}}1
    {ґ}{{\selectfont\char160}}1
    {І}{{\selectfont\char73}}1
    {Ї}{{\selectfont\char136}}1
    {Є}{{\selectfont\char153}}1
    {Ґ}{{\selectfont\char128}}1
}

\begin{document}
\begin{titlepage}

    \begin{table}[H]
        \centering
        \footnotesize
        \begin{tabular}{cc}
            \multirow{8}{*}{\includegraphics[scale=0.35]{bmstu.jpg}}
            & \\
            & \\
            & \textbf{Министерство науки и высшего образования Российской Федерации} \\
            & \textbf{Федеральное государственное бюджетное образовательное учреждение} \\
            & \textbf{высшего образования} \\
            & \textbf{<<Московский государственный технический} \\
            & \textbf{университет имени Н.Э. Баумана>>} \\
            & \textbf{(МГТУ им. Н.Э. Баумана)} \\
        \end{tabular}
    \end{table}

    \vspace{-2.5cm}

    \begin{flushleft}
        \rule[-1cm]{\textwidth}{3pt}
        \rule{\textwidth}{1pt}
    \end{flushleft}

    \begin{flushleft}
        \small
        ФАКУЛЬТЕТ
        \underline{<<Информатика и системы управления>>\ \ \ \ \ \ \ 
        \ \ \ \ \ \ \ \ \ \ \ \ \ \ \ \ \ \ \ \ \ \ \ \ \ \ \ \ \ \ \ 
    \ \ \ \ \ \ \ \ \ \ \ \ \ \ \ } \\
        КАФЕДРА
        \underline{<<Программное обеспечение ЭВМ и
        информационные технологии>>
        \ \ \ \ \ \ \ \ \ \ \ \ \ \ \ \ \ \ \ \ }
    \end{flushleft}

    \vspace{4cm}

    \begin{center}
        \textbf{Лабораторная работа № 20} \\ 
        \hfill
        
        \textbf{Формирование и модификация списков на Prolog} \\
        \vspace{0.5cm}
        \textbf{} \\
    \end{center}

    \vspace{4cm}

    \begin{flushleft}
        \begin{tabular}{ll}
            \textbf{Дисциплина} & Функциональное и логическое программирование \\
            \textbf{Студент} & Сиденко А.Г. \\
            \textbf{Группа} & ИУ7-63Б \\
            \textbf{Преподаватель} & Толпинская Н.Б., Строганов Ю.В.  \\
        \end{tabular}
    \end{flushleft}

    \vspace{4cm}

   \begin{center}
        Москва, 2020 г.
    \end{center}

\end{titlepage}

\textbf{Задание}

Используя хвостовую рекурсию, разработать, комментируя аргументы, эффективную программу, позволяющую:

\begin{enumerate}
\item Сформировать список из элементов числового списка, больших заданного значения;
\item Сформировать список из элементов, стоящих на нечетных позициях исходного списка (нумерация от 0);
\item Удалить заданный элемент из списка (один или все вхождения);
\item Преобразовать список в множество (можно использовать ранее разработанные процедуры).
\end{enumerate}

Убедиться в правильности результатов. 

Для одного из вариантов вопроса и 1-ого задания  составить таблицу, отражающую конкретный порядок работы системы



\hfill

\textbf{Программа}

\begin{lstlisting}
domains
  list = integer*
  num = integer
predicates
  over_number(list, num, list).
  odd_list(list, list).
  delete_number(list, num, list).
  member(num, list).
  multi_list(list, list).
clauses    
  over_number([], _, []):-!.
  over_number([X|Tail], Num, [X|List]):-
    X > Num,
    over_number(Tail, Num, List),!.
  over_number([_|Tail], Num, List):-
    over_number(Tail, Num, List).
    
  odd_list([], []):-!.
  odd_list([_], []):-!.
  odd_list([_, Second|Tail], [Second|OddTail]):-
    odd_list(Tail, OddTail).
    
  delete_number([], _, []):-!.
  delete_number([Num|Tail], Num, List):-
    delete_number(Tail, Num, List),!.
  delete_number([X|Tail], Num, [X|List]):-
    delete_number(Tail, Num, List).
    
  member(Num, [Num|_]):-!.
  member(Num, [_|Tail]):-
    member(Num, Tail).

  multi_list([], []):-!.
  multi_list([Num|Tail], List):-
    member(Num, Tail),
    multi_list(Tail, List),!.
  multi_list([Num|Tail], [Num|List]):-
    multi_list(Tail, List).
\end{lstlisting}

\textbf{Приведем таблицу для задания 1. }
\begin{lstlisting}
goal
  over_number([5,1,2,3], 2, List).
\end{lstlisting}

\begin{longtable}{|p{0.5cm}|p{5cm}|p{6cm}|p{5.5cm}|}
	\hline
 	№ шага & Состояние резольвенты & Сравниваемые термы; результат; подстановка, если есть  & Дальнейшие действия: прямой ход или откат \\ \hline
	1 & $over\_number$([5,1,2,3], 2, List) & По $over\_number$([5,1,2,3], 2, List)) ищется системой определение отношения (по имени предиката и списку (числу) аргументов) & Определение отношения найдено, заносится в стек $over\_number$([5,1,2,3], 2, List), прямой ход \\ \hline
	
	2 &5 > 2, $over\_number$([1,2,3], 2, List) & Начинает <<раскрываться>> правило, т.е. доказывается каждое целевое утверждение в теле правила последовательно слева направо
	X > Num, $over\_number$(Tail, Num, List) & Прямой ход\\ \hline
	
	3 & $over\_number$([1,2,3], 2, List) & 5>2 & Успешная унификация, переход к следующему целевому утверждению \\ \hline
	
	4 & $over\_number$([1,2,3], 2, List) & По $over\_number$([1,2,3], 2, List)) ищется системой определение отношения (по имени предиката и списку (числу) аргументов) & Определение отношения найдено, заносится в стек $over\_number$([1,2,3], 2, List), прямой ход \\ \hline
	
	5 &1 > 2, $over\_number$([2,3], 2, List) & Начинает <<раскрываться>> правило, т.е. доказывается каждое целевое утверждение в теле правила последовательно слева направо
	X > Num, $over\_number$(Tail, Num, List) & Прямой ход\\ \hline
	
	6 & $over\_number$([2,3], 2, List) & 1>2 & Неуспешная унификация, переход к следующему определению с таким именем \\ \hline
	
	7 & $over\_number$([2,3], 2, List) & Начинает <<раскрываться>> правило, т.е. доказывается каждое целевое утверждение в теле правила последовательно слева направо
	$over\_number$(Tail, Num, List) & Прямой ход\\ \hline
	
	8 & $over\_number$([2,3], 2, List) & По $over\_number$([2,3], 2, List)) ищется системой определение отношения (по имени предиката и списку (числу) аргументов) & Определение отношения найдено, заносится в стек $over\_number$([2,3], 2, List), прямой ход \\ \hline
	
	9 &2 > 2, $over\_number$([3], 2, List) & Начинает <<раскрываться>> правило, т.е. доказывается каждое целевое утверждение в теле правила последовательно слева направо
	X > Num, $over\_number$(Tail, Num, List) & Прямой ход\\ \hline
	
	10 & $over\_number$([3], 2, List) & 2>2 & Неуспешная унификация, переход к следующему определению с таким именем \\ \hline
	
	11 & $over\_number$([3], 2, List) & Начинает <<раскрываться>> правило, т.е. доказывается каждое целевое утверждение в теле правила последовательно слева направо
	$over\_number$(Tail, Num, List) & Прямой ход\\ \hline
	
	12 & $over\_number$([3], 2, List) & По $over\_number$([3], 2, List)) ищется системой определение отношения (по имени предиката и списку (числу) аргументов) & Определение отношения найдено, заносится в стек $over\_number$([3], 2, List), прямой ход \\ \hline
	
	13 &3 > 2, $over\_number$([], 2, List) & Начинает <<раскрываться>> правило, т.е. доказывается каждое целевое утверждение в теле правила последовательно слева направо
	X > Num, $over\_number$(Tail, Num, List) & Прямой ход\\ \hline
	
	14 & $over\_number$([], 2, List) & 2>2 & Успешная унификация, переход к следующему целевому утверждению \\ \hline
	
	15 & $over\_number$([], 2, List) & По $over\_number$([], 2, List)) ищется системой определение отношения (по имени предиката и списку (числу) аргументов) & Определение отношения найдено, заносится в стек $over\_number$([], 2, List), прямой ход \\ \hline
	
	16 &   & Унификация $over\_number$([], 2, List)  с $over\_number$([], $\_$, []) & Унификация успешна, List=[], отсечение, возврат из стека $over\_number$([], 2, List)\\ \hline	
	
	17 & Резольента пуста & Поочередно достаем из стека и подставляем X, List & List =[3], достаем из стека $over\_number$([3], 2, List), откат \\ \hline
	
	18 & Резольента пуста & Поочередно достаем из стека и подставляем X, List & List =[3], достаем из стека $over\_number$([2,3], 2, List), откат \\ \hline
	
	19 & Резольента пуста & Поочередно достаем из стека и подставляем X, List & List =[3], достаем из стека $over\_number$([1,2,3], 2, List), откат \\ \hline
	
	20 & Резольента пуста & Поочередно достаем из стека и подставляем X, List & List =[5,3], достаем из стека $over\_number$([5,1,2,3], 2, List), откат \\ \hline
	
	21 & Резольента пуста &  & Стек пуст. Вывод результата. Завершение программы.  \\ \hline
\end{longtable}

\hfill

\textbf{Вывод}

Эффективный способ организации рекурсии --  хвостовая рекурсия. Эффективность рекурсивной процедуры повышается благодаря отсечению неперспективных путей поиска решения. Используя <<!>> -- отсечение. Которое сократит количество выполняемых унификаций для достижения максимальной эффективности работы системы. 

\hfill

\textbf{Ответы на вопросы}

\begin{enumerate} 
\item Как организуется хвостовая рекурсия в Prolog? 

Хвостовая рекурсия: Для ее осуществления рекурсивный вызов определяемого предиката должен быть последней подцелью в теле рекурсивного правила и к моменту рекурсивного вызова не должно остаться точек возврата (непроверенных альтернатив). 

\item Какое первое состояние резольвенты?

Вопрос. 

\item Каким способом можно разделить список на части, какие, требования к частям?

В Prolog существует более общий способ доступа к элементам списка. Для этого используется метод разбиения списка на начало и остаток. Для этого используется вертикальная черта (|) за последним элементом начала. 

Начало списка -- это группа первых элементов, не менее одного. Остаток списка -- обязательно список (может быть пустой), всегда один. 

\item Как выделить за один шаг первые два подряд идущих элемента списка? Как выделить 1-й и 3-й элемент за один шаг?

Два подряд идущих:

\begin{lstlisting}
[First, Second|Tail]
\end{lstlisting}

1-й и 3-й:

\begin{lstlisting}
[First, _, Third|Tail]
\end{lstlisting}

\item Как формируется новое состояние резольвенты?

Резольвента - текущая цель, существующая на любой стадии вычислений. Резольвенты порождаются целью и каким-либо правилом или фактом, которые просматриваются последовательно сверху вниз. Если резольвента существует при наиболее общей унификации, она вычисляется. Если пустая резольвента с помощью такой стратегии не найдена, то ответ на вопрос отрицателен.

\item Когда останавливается работа системы? Как это определяется на формальном уровне?

Когда стек пуст. 

\end{enumerate}
 
\end{document}