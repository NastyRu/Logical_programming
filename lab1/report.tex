\documentclass[a4paper,14pt]{extreport} % формат документа

\usepackage{amsmath}
\usepackage{cmap} % поиск в ПДФ
\usepackage[T2A]{fontenc} % кодировка
\usepackage[utf8]{inputenc} % кодировка исходного текста
\usepackage[english,russian]{babel} % локализация и переносы
\usepackage[left = 2cm, right = 1cm, top = 2cm, bottom = 2 cm]{geometry} % поля
\usepackage{listings}
\usepackage{graphicx} % для вставки рисунков
\usepackage{amsmath}
\usepackage{float}
\usepackage{multirow}
\graphicspath{{pictures/}}
\DeclareGraphicsExtensions{.pdf,.png,.jpg}
\newcommand{\anonsection}[1]{\section*{#1}\addcontentsline{toc}{section}{#1}}

\lstset{ %
	language=Prolog,                % Язык программирования 
	numbers=left,                   % С какой стороны нумеровать          
	frame=single,                    % Добавить рамку
}

\begin{document}
\begin{titlepage}

    \begin{table}[H]
        \centering
        \footnotesize
        \begin{tabular}{cc}
            \multirow{8}{*}{\includegraphics[scale=0.35]{bmstu.jpg}}
            & \\
            & \\
            & \textbf{Министерство науки и высшего образования Российской Федерации} \\
            & \textbf{Федеральное государственное бюджетное образовательное учреждение} \\
            & \textbf{высшего образования} \\
            & \textbf{<<Московский государственный технический} \\
            & \textbf{университет имени Н.Э. Баумана>>} \\
            & \textbf{(МГТУ им. Н.Э. Баумана)} \\
        \end{tabular}
    \end{table}

    \vspace{-2.5cm}

    \begin{flushleft}
        \rule[-1cm]{\textwidth}{3pt}
        \rule{\textwidth}{1pt}
    \end{flushleft}

    \begin{flushleft}
        \small
        ФАКУЛЬТЕТ
        \underline{<<Информатика и системы управления>>\ \ \ \ \ \ \ 
        \ \ \ \ \ \ \ \ \ \ \ \ \ \ \ \ \ \ \ \ \ \ \ \ \ \ \ \ \ \ \ 
    \ \ \ \ \ \ \ \ \ \ \ \ \ \ \ } \\
        КАФЕДРА
        \underline{<<Программное обеспечение ЭВМ и
        информационные технологии>>
        \ \ \ \ \ \ \ \ \ \ \ \ \ \ \ \ \ \ \ \ }
    \end{flushleft}

    \vspace{4cm}

    \begin{center}
        \textbf{Лабораторная работа № 11} \\ 
        \hfill
        
        \textbf{Среда Visual Prolog 5.2} \\
        \vspace{0.5cm}
        \textbf{} \\
    \end{center}

    \vspace{4cm}

    \begin{flushleft}
        \begin{tabular}{ll}
            \textbf{Дисциплина} & Функциональное и логическое программирование \\
            \textbf{Студент} & Сиденко А.Г. \\
            \textbf{Группа} & ИУ7-63Б \\
            \textbf{Преподаватель} & Толпинская Н.Б., Строганов Ю.В.  \\
        \end{tabular}
    \end{flushleft}

    \vspace{4cm}

   \begin{center}
        Москва, 2020 г.
    \end{center}

\end{titlepage}

\textbf{Задание}

Запустить среду Visual Prolog5.2. Настроить утилиту TestGoal. Запустить тестовую программу, проанализировать реакцию системы и множество ответов. Разработать свою программу -- <<Телефонный справочник>>. Протестировать работу программы. 

\textbf{Программа <<Телефонный справочник>>}

\begin{lstlisting}
predicates
  multi abonent(string, string).
clauses
  abonent(ellen, "111111").
  abonent(ellen, "777777").
  abonent(john, "222222").
  abonent(tom, "333333").
  abonent(eric, "444444").
  abonent(eric, "888888").
  abonent(eric, "999999").
  abonent(mark, "555555").
  abonent(bill, "666666").
goal
  Name = mark,
  write(Name, " numbers: "), nl,
  abonent(Name, Number).
\end{lstlisting}

\textbf{Примеры работы}

\begin{enumerate}
\item Если имени нет в телефонной книге

\includegraphics{ex1}

\item Если имя встречается один раз в телефонной книге

\includegraphics{ex2}

\newpage

\item Если имя встречается несколько раз в телефонной книге

\includegraphics{ex3}
\end{enumerate}

\textbf{Ответы на вопросы}

что собой представляет программа на Prolog, какова ее структура. Как она реализуется, как формируются результаты работы программы. 

\textbf{Программа на Prolog представляет собой:} базу знаний и вопрос. База знаний содержит истинностные знания, используя которые программа выдает ответ на запрос. 

Основным элементом языка является терм. Терм – это: константа, переменная, составной терм. С помощью термов и более сложных конструкций языка Prolog – фактов и правил <<описываются>> знания о предметной области, т.е. база знаний. Используя базу знаний, система Prolog будет делать логические выводы, отвечая на наши вопросы. 


\textbf{Структура программы }
\begin{enumerate}
\item раздел \textbf{constants} содержит определения констант, необязательный раздел.
\item раздел \textbf{domains} содержит определения доменов, которые описывают различные классы объектов, используемых в программе.
\item раздел \textbf{database} содержит утверждения базы данных, которые являются предикатами внутренней базы данных. Если программа такой базы данных не требует, то этот раздел может быть опущен.
\item раздел \textbf{predicates} служит для описания используемых программой предикатов.
\item в раздел \textbf{clauses} заносятся факты и правила, известные априорно (утверждения). Это данные, необходимых для работы программы.
\item в разделе \textbf{goal} формулируется назначение создаваемой программы. Это раздел описания цели. Составными частями при этом могут являться некие подцели, из которых формируется единая цель программы.
\end{enumerate}

С помощью подбора ответов на запросы он (Prolog, программа) извлекает хранящуюся (известную в программе) информацию. Одной из особенностей Prolog является то, что при поиске ответов на вопрос, он рассматривает альтернативные варианты и находит все возможные решения (методом проб и ошибок) -- множества значений переменных, при которых на поставленный вопрос можно ответить -- <<да>>.

Поиск содержательного ответа на поставленный вопрос, с помощью имеющейся базы знаний, фактически заключается в поиске нужного знания, но какое знание понадобится – заранее неизвестно. Этот поиск осуществляется формально с помощью механизма унификации. Упрощенно, процесс унификации можно представить как формальный процесс сравнивания терма вопроса с очередным термом знания. При этом, знания по умолчанию просматриваются сверху вниз. В процессе сравнивания для переменных «подбираются», исходя из базы знаний, значения или подтверждается истинность вопроса. 

\end{document}